Die vorliegende Dokumentation beschreibt die Entwicklung einer Aufgabenmanager-Anwendung namens TaskMaster, die im Rahmen des Moduls "Infrastruktur" an der Hochschule entwickelt wurde. Diese Anwendung basiert auf dem bewährten LEMP-Stack und nutzt Redis als Caching-Mechanismus sowie Ajax für asynchrone Anfragen. Die Dokumentation gibt einen umfassenden Überblick über die Architektur, die Funktionalitäten und die Umsetzung des TaskMasters.Die vorliegende Dokumentation beschreibt die Entwicklung einer Aufgabenmanager-Anwendung namens TaskMaster, die im Rahmen des Moduls "Infrastruktur" an der Hochschule entwickelt wurde. Diese Anwendung basiert auf dem bewährten LEMP-Stack und nutzt Redis als Caching-Mechanismus sowie Ajax für asynchrone Anfragen. Die Dokumentation gibt einen umfassenden Überblick über die Architektur, die Funktionalitäten und die Umsetzung des TaskMasters.

\section{Zielsetzung}

Das Hauptziel dieses Projekts war die Entwicklung einer effizienten und benutzerfreundlichen Aufgabenmanager-Anwendung, die es Benutzern ermöglicht, ihre Aufgaben zu erstellen, zu bearbeiten, zu verfolgen und gemeinsam zu arbeiten. Dabei sollte die Anwendung eine hohe Leistung und eine nahtlose Benutzererfahrung bieten.

\section{Funktionalitäten des TaskMasters}

Der TaskMaster bietet den Benutzern eine Vielzahl von Funktionen, darunter:

\begin{itemize}
    \item Erstellung und Bearbeitung von Aufgaben mit individuellen Attributen wie Titel, Beschreibung, Priorität und Frist.
    \item Zuweisung von Aufgaben an andere Benutzer zur Zusammenarbeit und Koordination.
    \item Verfolgung des Fortschritts der Aufgaben, Markierung als erledigt und Anzeige der verbleibenden Aufgaben.
    \item Benutzerregistrierung und Authentifizierung.
\end{itemize}



